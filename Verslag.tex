\documentclass[10pt,a4paper]{article}
\usepackage[utf8]{inputenc}
\usepackage{amsmath}
\usepackage{amsfonts}
\usepackage{amssymb}
\usepackage{graphicx}

\title{Het getal $\pi$}
\author{
	Lars Folkersma\\
	5543800\\
	\and
	Paul van Grol \\
	5528909
	\and
	Artuur Oerlemans \\
	.......}

\begin{document}
\maketitle
\newpage

\section{Bewijs}
\subsection{Argumenten van complexe getallen}
Het argument van een complex getal $ x+yi $ wordt gegeven door $ arctan(y/x) $ en het argument van een complex getal $ x-yi $ wordt gegeven door $ -arctan(y/x) $, onder de voorwaarde dat $ x>0 $.
Verder geldt dat als we complexe getallen vermenigvuldigen, dat het argument van het resultaat gelijk is aan de som van de argumenten.
\subsection{Voorbeeld}
We kunnen nu dus de getallen $ 2+i $ en $ 3+i $ nemen, die als argument respectievelijk $ arctan(1/2) $ en $ arctan(1/3) $ hebben. Het product van de twee getallen is gelijk aan $ 5 + 5i $, waarvan het argument dus gelijk is aan $ arctan(5/5) = arctan(1) = \pi / 4 $. Dit argument is tegelijk ook gelijk aan de som van de argumenten, ofwel $ \pi / 4 = arctan(1/2)+arctan(1/3) $
\subsection{Formule van Machin}
Om de formule van Machin te bewijzen, maken we gebruik van het volgende:
$ (1+i) (5-i)^4 (239+i) = 228488 $. Als we hier de argumenten van nemen, dan zien we het volgende: $ arctan(1) - 4 arctan(1/5) + arctan(1/239) = 0 $, ofwel $ \pi/4 = 4 arctan(1/5) - arctan(1/239) $; de formule van Machin.
\end{document}
